\documentclass[oneside,final,14pt]{extreport}
\usepackage[utf8]{inputenc}
\usepackage{setspace}
\singlespacing
\usepackage[russianb]{babel}
\usepackage{vmargin}
\usepackage{amsthm}
\theoremstyle{plain}
\usepackage{amsfonts}
\usepackage{mathtools}
\usepackage{graphicx}
\usepackage{float}
\setpapersize{A4}
\setmarginsrb{2cm}{1.5cm}{1cm}{1.5cm}{0pt}{0mm}{0pt}{13mm}
\usepackage{indentfirst}
\parindent=1cm
\sloppy
\begin{document}



\tableofcontents
\chapter{Алгебраическое введение}
\section{Кольцо целых алгебраических}
Чтобы понять что тут вообще происходит, нужно немного алгебры.\\
Основное поле - $\mathbb{Q}$. Корни полиномов с коэффициентами из 
$\mathbb{Q}$ называют алгебраическими над $\mathbb{Q}$. Алгебраическое число называется целым алгебраическим, если его минимальный многочлен $g(x) \in \mathbb{Z}[x]$. Введем множество целых алгебраических поля 
$\mathbb{Q}(\alpha)$
$$\mathfrak{O}_{\mathbb{Q}(\alpha)}=\{x \;|\; x \in \mathbb{Q}(\alpha
)\; \& \; x\text{ - целое алгебраическое}  \}$$

\newtheorem*{theorem*}{T(?)}
\begin{theorem*}
 $\mathfrak{O}_{\mathbb{Q}(\alpha)}$ - кольцо. Кроме того  $\mathbb{Z}[\alpha] \subseteq \mathfrak{O}_{\mathbb{Q}(\alpha)}$ - подкольцо.
\end{theorem*}


\begin{theorem*}
$K$ - поле, $R \subset K$ - подкольцо с однозначным разложением на множители. Тогда $R$ содержит корни всех неприводимых унитарных многочленов из $R[x]$
\end{theorem*}

Нас будет интересовать подкольцо $\mathbb{Z}[\alpha]$ c однозначным разложением на множители. Но в каких случаях будет получаться это однозначное разложение? Теорема гарантирует однозначное разложение в 
$\mathfrak{O}_{\mathbb{Q}(\alpha)}$. Если $\mathbb{Z}[\alpha]=\mathfrak{O}_{\mathbb{Q}(\alpha)}$ то все ок. Но это не всегда так.


\newtheorem*{remark*}{Пример}
\begin{remark*}
Рассмотрим $\gamma=-\frac{1}{2}+\frac{\sqrt{5}}{2} \notin \mathbb{Z}[\sqrt{5}]$.
Но $\gamma$ корень $x^2+x-1 \in \mathbb{Z}[x]$, значит $\gamma \in \mathfrak{O}_{\mathbb{Q}(\sqrt{5})}$
\end{remark*}

\begin{theorem*}
$\forall \beta \in \mathfrak{O}_{\mathbb{Q}(\alpha)} \rightarrow \beta \cdot f'(\alpha) \in \mathbb{Z}[\alpha]$, где $f'(x)$ - минимальный многочлен $\alpha$
\end{theorem*}

\section{Норма}

\chapter{Специальный метод решета числового поля}
Пусть нужно факторизовать число $$n=r^t-s$$
Нам нужно построить поле по некоторому полиному $f(x)$. Выберем d - степень полинома.
$$r^t-s\equiv0 \; mod \; n$$
$$r^t\equiv s \; mod \; n$$
Положим $k=\lceil\frac{t}{d}\rceil$ и домножим обе части на $r^{kd}$
$$r^{t+kd}\equiv sr^{kd} \; mod \; n$$
$$r^{kd}\equiv sr^{kd-t} \; mod \; n$$
Положим $m=r^{k}$ и $c=sr^{kd-t}$
$$m^{d}\equiv c \; mod \; n$$
В качестве полинома возьмем
$$f(x)=x^d-c$$
Построим поле
$$^{\mathbb{Q}[x]}/_{(f)} \cong \mathbb{Q}(\alpha)\text{, где $\alpha$ - корень $f(x)$}$$
Что же делать дальше?







\begin{thebibliography}{0}




\bibitem{Name} Text







\end{thebibliography}
\appendix
\chapter{Приложение}\label{AppendixA}

\end{document}