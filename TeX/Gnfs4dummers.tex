\documentclass[oneside,final,14pt]{extreport}
\usepackage[utf8]{inputenc}
\usepackage{setspace}
\singlespacing
\usepackage[russianb]{babel}
\usepackage{vmargin}
\usepackage{amsthm}
\theoremstyle{plain}
\usepackage{amsfonts}
\usepackage{mathtools}
\usepackage{graphicx}
\usepackage{mathtools}
\usepackage{float}
\setpapersize{A4}
\setmarginsrb{2cm}{1.5cm}{1cm}{1.5cm}{0pt}{0mm}{0pt}{13mm}
\usepackage{indentfirst}
\parindent=1cm
\sloppy
\begin{document}



\tableofcontents
\chapter{Алгебраическое введение}
\section{Кольцо целых алгебраических}
Чтобы понять что тут вообще происходит, нужно немного алгебры.\\
Основное поле - $\mathbb{Q}$. Корни полиномов с коэффициентами из 
$\mathbb{Q}$ называют алгебраическими над $\mathbb{Q}$. Алгебраическое число называется целым алгебраическим, если его минимальный многочлен $g(x) \in \mathbb{Z}[x]$. Введем множество целых алгебраических поля 
$\mathbb{Q}(\alpha)$
$$\mathfrak{O}_{\mathbb{Q}(\alpha)}=\{x \;|\; x \in \mathbb{Q}(\alpha
)\; \& \; x\text{ - целое алгебраическое}  \}$$

\newtheorem*{theorem*}{T(?)}
\begin{theorem*}
 $\mathfrak{O}_{\mathbb{Q}(\alpha)}$ - кольцо. Кроме того  $\mathbb{Z}[\alpha] \subseteq \mathfrak{O}_{\mathbb{Q}(\alpha)}$ - подкольцо.
\end{theorem*}


\begin{theorem*}
$K$ - поле, $R \subset K$ - подкольцо с однозначным разложением на множители. Тогда $R$ содержит корни всех неприводимых унитарных многочленов из $R[x]$
\end{theorem*}

Нас будет интересовать подкольцо $\mathbb{Z}[\alpha]$ c однозначным разложением на множители. Но в каких случаях будет получаться это однозначное разложение? Теорема гарантирует однозначное разложение в 
$\mathbb{Z}[\alpha]$, если $\mathbb{Z}[\alpha]=\mathfrak{O}_{\mathbb{Q}(\alpha)}$. Но это не всегда так. Что делать когда это не так?


\newtheorem*{remark*}{Пример}
\begin{remark*}
Рассмотрим $\gamma=-\frac{1}{2}+\frac{\sqrt{5}}{2} \notin \mathbb{Z}[\sqrt{5}]$.
Но $\gamma$ корень $x^2+x-1 \in \mathbb{Z}[x]$, значит $\gamma \in \mathfrak{O}_{\mathbb{Q}(\sqrt{5})}$
\end{remark*}

\begin{theorem*}
$\forall \beta \in \mathfrak{O}_{\mathbb{Q}(\alpha)} \rightarrow \beta \cdot f'(\alpha) \in \mathbb{Z}[\alpha]$, где $f'(x)$ - минимальный многочлен $\alpha$
\end{theorem*}



\section{Норма}


Вспомним, что $\mathbb{Q}(\alpha)$ - в.п. над $\mathbb{Q}$. Введем в этом пространстве линейный оператор, соответствующий некоторому элементу $\beta \in \mathbb{Q}(\alpha)$

\[
\mathcal{H}_{\beta}: \;\mathbb{Q}(\alpha) \xrightarrow[x \mapsto \beta x]{} \mathbb{Q}(\alpha)
\]

\newtheorem*{def*}{Опр}
\begin{def*}
Норма $\beta \in \mathbb{Q}(\alpha) $ определяется следующим равенством
$$\mathcal{N}(\beta)=det(\mathcal{H}_{\beta})$$
\end{def*}


Как вычислять эту норму для произвольного элемента? Пока не ясно. В алгоритме говорится, что норму элемента вида $\beta=a+b \alpha$ можно вычислить по формуле
$$
\mathcal{N}(\beta)=F(a,b) = b^{degf}f(\frac{a}{b}) \text{, где } f \text{ - полином степени порождающий поле.}
$$ 
Откуда эта формула? Как ее получить из определения? Нужно разбираться. Но пока будем считать, что все норм.


\begin{remark*}
Попробуем вычислить норму $\beta=a+b\sqrt{5}$ в $\mathbb{Q}(\sqrt{5})$.\\
Найдем $\mathcal{H}_{\beta}=\begin{pmatrix} x_1 & x_2 \\ x_3 & x_4 \end{pmatrix}$.
По определению эта матрица соответствует оператору, который $x$ отправляет в $\beta x$. Подействуем этой матрицей на элементы $1$ и $\sqrt{5}$. Им соответствуют векторы $(1,0)$ и $(0,1)$.
$$\beta \cdot 1=\beta=a+b\sqrt{5} \; \Rightarrow 
\begin{pmatrix} x_1 & x_2 \\ x_3 & x_4 \end{pmatrix}
\begin{pmatrix} 1 \\ 0\end{pmatrix}=
\begin{pmatrix} x_1 \\ x_3\end{pmatrix}=
\begin{pmatrix} a \\ b\end{pmatrix}$$
$$
\beta \cdot \sqrt{5}=5b+a\sqrt{5} \; \Rightarrow 
\begin{pmatrix} x_1 & x_2 \\ x_3 & x_4 \end{pmatrix}
\begin{pmatrix} 0 \\ 1\end{pmatrix}=
\begin{pmatrix} x_2 \\ x_4\end{pmatrix}=
\begin{pmatrix} 5b \\ a\end{pmatrix}
$$
Получили матрицу $\begin{pmatrix} a & 5b \\ b & a \end{pmatrix}$. Делаем вывод, что $\mathcal{N}(\beta)=det(\mathcal{H}_{\beta})=a^2-5b^2$.
Попробуем вычислить по формуле. Поле $\mathbb{Q}(\sqrt{5})$ построено с помощью полинома $f(x)=x^2-5$.
$$\mathcal{N}(\beta)=F(a,b) = b^2f(\frac{a}{b})=b^2 ((\frac{a}{b})^2-5)=a^2-5b^2$$
Магия...
\end{remark*}
\chapter{Специальный метод решета числового поля}
Пусть нужно факторизовать число $$n=r^t-s$$
Нам нужно построить поле по некоторому полиному $f(x)$. Выберем d - степень полинома.
$$r^t-s\equiv0 \; mod \; n$$
$$r^t\equiv s \; mod \; n$$
Положим $k=\lceil\frac{t}{d}\rceil$ и домножим обе части на $r^{kd}$
$$r^{t+kd}\equiv sr^{kd} \; mod \; n$$
$$r^{kd}\equiv sr^{kd-t} \; mod \; n$$
Положим $m=r^{k}$ и $c=sr^{kd-t}$
$$m^{d}\equiv c \; mod \; n$$
В качестве полинома возьмем
$$f(x)=x^d-c$$
Построим поле
$$^{\mathbb{Q}[x]}/_{(f)} \cong \mathbb{Q}(\alpha)\text{, где $\alpha$ - корень $f(x)$}$$
Что же делать дальше?







\begin{thebibliography}{0}




\bibitem{Name} Text







\end{thebibliography}
\appendix
\chapter{Приложение}\label{AppendixA}

\end{document}